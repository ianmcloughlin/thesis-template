\chapter*{Abstract}

Writing a compelling abstract for a Ph.D. thesis requires conciseness, clarity, and the ability to convey the significance of your research. Here are some tips to help you craft an effective abstract.
(1) Clearly State the Problem:
Begin by clearly stating the problem or question your research addresses.
Be concise and specific about the problem you are investigating.
(2) Highlight the Objective:
Clearly state the main objective of your research.
What are you contributing to the field of study?
Make it evident how your work fits into the broader context.
(3) Provide a Brief Overview of Methods:
Mention the key methods used in your research.
Briefly explain the tools or frameworks you used to address the problem.
However, avoid going into excessive detail.
(4) Present Key Results:
Summarize the main findings and results of your research. 
Highlight any breakthroughs, novel insights, or contributions your work has made to the field.
(5) Contextualize the Significance:
Communicate the significance and relevance of your research. 
Explain how your findings contribute to and address gaps in the field.
(6) Use Concise and Accessible Language:
Write in a clear, concise, and accessible language.
Avoid unnecessary jargon that may be unclear to readers outside your specific subfield of study.
Remember that the abstract serves as a short summary of your entire Ph.D. thesis.
It provides readers with a quick overview of your research. Striking the right balance between conciseness and informativeness is key.
Put your best foot forward without overstating your findings.
