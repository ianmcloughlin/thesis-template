%!TEX root = ../thesis.tex

\chapter*{Abstract}


Writing a compelling abstract for a Ph.D. thesis requires conciseness, clarity, and the ability to convey the significance of your research. Here are some tips to help you craft an effective abstract.
(1) Clearly State the Problem:
Begin by clearly stating the problem or question your research addresses.
Be concise and specific about the problem you are investigating.
(2) Highlight the Objective:
Clearly state the main objective of your research.
What are you contributing to the field of study?
Make it evident how your work fits into the broader context.
(3) Provide a Brief Overview of Methods:
Mention the key methods used in your research.
Briefly explain the tools or frameworks you used to address the problem.
However, avoid going into excessive detail.
(4) Present Key Results:
Summarize the main findings and results of your research. 
Highlight any breakthroughs, novel insights, or contributions your work has made to the field.
(5) Contextualize the Significance:
Communicate the significance and relevance of your research. 
Explain how your findings contribute to and address gaps in the field.
(6) Use Concise and Accessible Language:
Write in a clear, concise, and accessible language.
Avoid unnecessary jargon that may be unclear to readers outside your specific subfield of study.
Remember that the abstract serves as a short summary of your entire Ph.D. thesis.
It provides readers with a quick overview of your research. Striking the right balance between conciseness and informativeness is key.
Put your best foot forward without overstating your findings.




\chapter*{About the Author}
When writing about yourself, highlight your academic background, research interests, accomplishments, and any relevant experiences.
Here are some tips to help you craft an effective blurb.
(1) Start with a Strong Opening:
Begin with a concise and engaging statement that captures the reader's attention. For example, you might start with a brief description of your academic journey or research focus.
(2) Highlight Your Academic Background:
Provide a brief overview of your academic qualifications, including your degree(s) and any relevant academic honors or awards you have received.
(3) Emphasize Your Research Interests:
Clearly articulate your research interests and the topics you are passionate about. Briefly mention any specific areas of mathematics or related fields that you have focused on in your thesis work.
(4) Showcase Your Accomplishments:
Highlight any significant achievements or contributions you have made in your field. This could include publications, conference presentations, research projects, or collaborations.
(5) Include Relevant Experiences:
Mention any relevant academic or professional experiences that have shaped your research interests and expertise. This could include internships, research assistantships, teaching positions, or involvement in academic societies.