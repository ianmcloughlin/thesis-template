%!TEX root = ../thesis.tex

\chapter{Introduction}
\label{section:introduction}

In the introduction, you should describe what your thesis is about, how the
thesis is organised, and what the reader can expect as they read down through
it.
The most important aspect of the introduction is to set the context for the
rest of the thesis.
You should sign-post for the reader the most important parts of your work and
where they appear in the document.
In \LaTeX, you should refer to sections, tables, and figures using commands
rather than specifying a page or saying ``the table below''. The \texttt{ref}
command will keep track of changes to the layout to the document. So, for
example, I can just refer to Section \ref{section:literature} rather than
worrying where that might move to in future.
Note to refer to an item in the bibliography, you use the \texttt{cite} command,
which should generally be placed after a \texttt{tilde}(\texttt{\~}) rather than
a space~\cite{einstein}.


\section{What Should be Included Here}
The introduction chapter of a Ph.D. thesis serves as the gateway to your research and sets the stage for the reader to understand the context, significance, and objectives of your study.
Here's a comprehensive guide on what to include in the introduction chapter.

\subsection{Background and Context}
Provide an overview of the broader field of study and the specific research area your thesis addresses.
Discuss the historical background, key concepts, theories, and previous research relevant to your topic.

\subsection{Research Problem and Motivation}
Clearly articulate the research problem or question your thesis aims to address.
Explain why this problem is significant and worthy of investigation.
Discuss any gaps or limitations in existing literature that your research seeks to fill.

\subsection{Objectives and Research Questions}
State the objectives of your research and the specific research questions you seek to answer.
These objectives should be clear, specific, and aligned with the overall purpose of your study.

\subsection{Scope and Limitations}
Define the scope of your research by outlining what is included and excluded from your study.
Discuss any limitations or constraints that may impact the interpretation or generalizability of your findings.

\subsection{Conceptual Framework or Theoretical Framework}
If applicable, introduce the conceptual framework or theoretical framework that underpins your research.
Explain the theoretical perspectives, models, or frameworks you will use to guide your analysis and interpretation.

\subsection{Methodology}
Provide an overview of the research methodology and approach you have adopted.
Discuss the research design, data collection methods, analytical techniques, and any other procedures used to conduct your study.

\subsection{Significance and Contributions}
Clearly articulate the significance of your research and the potential contributions it makes to the field.
Explain how your study advances knowledge, addresses gaps in the literature, or has practical implications.

\subsection{Organizational Structure}
Outline the structure of your thesis by briefly describing the contents of each chapter.
Provide a roadmap that helps the reader navigate through your thesis and understand the sequence of your argumentation.

\subsection{Literature Review}
While the main literature review may be presented in a separate chapter, briefly summarize the key literature that informs your research in the introduction.
Highlight the most relevant theories, concepts, and empirical studies that provide context for your study.

\subsection{Engage the Reader}
Write in a clear, engaging, and concise manner to capture the reader's interest.
Use compelling language and examples to draw the reader into the topic and motivate them to continue reading.
Remember, the introduction chapter sets the tone for your entire thesis and should provide a comprehensive overview of your research topic, objectives, methodology, and significance.
It should be well-structured, focused, and persuasive, laying the foundation for the reader to understand and appreciate the rest of your work.

